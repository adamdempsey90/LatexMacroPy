\section{The deposited torque}

The equations of motion for the total angular momentum density and surface density are,

\begin{equation} \label{eq:tot_ang}
\pderiv{t} \left( \Sigma r^2 \Omega \right) + \div \left( \Sigma  v_r r^2 \Omega - \vec{\Pi_{r\phi}} \right) = - \Sigma r \del \Phi ,
\end{equation}
\begin{equation} \label{eq:tot_sigma}
\ppderiv{\Sigma}{t} + \div \left( \Sigma \vec{v} \right) .
\end{equation}

Our goal is to determine the evolution of the azimuthally-averaged surface density, $\avg{\Sigma}$. In particular, we are interested in the \emph{steady-state} surface density as a function of the disk's viscosity, mass flux, and the mass of the embedded planet. The evolution of the averaged surface density follows,
\begin{equation} \label{eq:avg_sigma}
\pderiv{t}\avg{2 \pi r \Sigma} + \pderiv{r} \avg{2 \pi r \Sigma v_r}  = 0 .
\end{equation}

We will obtain an expression for the radial mass flux, $\dot{M} = \avg{-2 \pi r \Sigma v_r}$, that depends on the rate at which angular momentum is added and removed to the axisymmetric flow of the disk. Define the \emph{mean} component of a quantity, $Q_1(r,\phi)$, as $\avg{Q_1}(r) = \int_{-\pi}^{\pi} \frac{d\phi'}{2 \pi} \, Q_1(r,\phi')$, and the \emph{wave} component as $Q_1'(r,\phi) = Q_1(r,\phi) - \avg{Q_1}(r)$. The azimuthal average of a wave term is by definition zero, but the average of a product of wave terms need not be zero, i.e $\avg{ Q_1 '} = 0$, but $\avg{ Q_1' Q_2'} \neq 0$. This allows us to split the average of a product of variables into a mean component and a wave component as, $\avg{Q_1 Q_2} = \avg{Q_1} \avg{Q_2} + \avg{ Q_1' Q_2'}$, respectively. 

Starting from the average angular momentum density,
\begin{equation} \label{eq:tot_avg_ang}
\pderiv{t} \avg{ \Sigma r^2 \Omega} + \frac{1}{r} \pderiv{r} \avg{ r\Sigma v_r r^2 \Omega - r^2 \Pi_{r\phi} }= \avg{ - \Sigma \ppderiv{\Phi}{\phi}},
\end{equation}
and using \eqref{eq:avg_sigma} we derive an expression for the mass flux,
\begin{equation} \label{eq:mass_flux}
\avg{r \Sigma v_r} = \left( \ppderiv{\avg{r^2 \Omega}}{r} \right)^{-1} \times \left[ -\pderiv{r} \left( \nu \avg{\Sigma} r^3 \pderiv{r} \avg{\Omega} \right) + r \avg{-\Sigma \ppderiv{\Phi}{\phi}} - \pderiv{r} \left( r F_w \right) - r\pderiv{t}\avg{ \Sigma' r^2 \Omega'} - r\avg{\Sigma} \pderiv{t}\avg{r^2 \Omega} \right] ,
\end{equation}
where the wave angular momentum flux, $F_w$, is 
\begin{equation} \label{eq:wave_flux}
    F_w = \avg{\Sigma} \avg{v_r' r^2 \Omega'} + \avg{v_r} \avg{ \Sigma' r^2 \Omega'} - \nu \avg{ \Sigma' \ppderiv{v_r'}{\phi}} - \nu \avg{ \Sigma'r^2 \ppderiv{\Omega'}{r}} .
\end{equation}
The first term in the parenthesis on the right-hand-side of \eqref{eq:mass_flux} is the classic viscous angular momentum flux . In the absence of the planet, any wave structure, and fixing the specific angular momentum profile, the disk viscously spreads as in \cite{1974MNRAS.168..603L}, i.e,
\begin{equation}
\pderiv{t} \avg{\Sigma} + \frac{1}{r} \pderiv{r} \left[ \left( \ppderiv{\avg{r^2 \Omega}}{r} \right)^{-1}\pderiv{r} \left( - \nu \avg{\Sigma} r^3 \pderiv{r} \avg{\Omega} \right) \right] = 0 .
\end{equation}
The second term is the rate at which the planet adds angular momentum into the waves launched at Linblad resonances \citep{1979ApJ...233..857G,1980ApJ...241..425G}. We define this term as the \emph{excited} torque,
\begin{equation}
\Lambda_\text{ex} = \avg{ - \Sigma \ppderiv{\Phi}{\phi} } ,
\end{equation}
In a disk with a fixed specific angular momentum profile and neglecting any wave structure we recover the familiar 1D disk evolution equation \citep{1986ApJ...309..846L},
\begin{equation} \label{eq:1d_no_wave}
\pderiv{t} \avg{\Sigma} + \frac{1}{r} \pderiv{r} \left[ \left( \ppderiv{\avg{r^2 \Omega}}{r} \right)^{-1} \left( \pderiv{r} \left( - \nu \avg{\Sigma} r^3 \pderiv{r} \avg{\Omega} \right) + \Lambda_\text{ex} \right) \right] = 0 .
\end{equation}

Taking into account the action of wave flux in \eqref{eq:mass_flux}, reduces (enhances) the excited torque in regions where the wave flux increases (decreases) across a ring of gas in the radial direction. We call this modified torque the \emph{deposited} torque, 
\begin{equation}
\Lambda_\text{dep} = \Lambda_{ex} - \delr{F_w} - \pderiv{t} \avg{ \Sigma' r^2 \Omega'} .
\end{equation}
A more general form of \eqref{eq:1d_no_wave} that includes the action of wave transport is now,
\begin{equation}
\pderiv{t} \avg{\Sigma} + \frac{1}{r} \pderiv{r} \left[ \left( \ppderiv{\avg{r^2 \Omega}}{r} \right)^{-1} \left( \pderiv{r} \left( - \nu \avg{\Sigma} r^3 \pderiv{r} \avg{\Omega} \right) + \Lambda_\text{dep} \right) \right] = 0 .
\end{equation}
Finally, if the specific angular momentum changes in time,
\begin{equation}
\pderiv{t} \avg{\Sigma} + \frac{1}{r} \pderiv{r} \left[ \left( \ppderiv{\avg{r^2 \Omega}}{r} \right)^{-1} \left( \pderiv{r} \left( - \nu \avg{\Sigma} r^3 \pderiv{r} \avg{\Omega} \right) + \Lambda_\text{dep}  - r \pderiv{t} \avg{r^2 \Omega} \right)\right] = 0 .
\end{equation}

There are four important timescales governing the evolution of the disk; the orbital timescale, $\tau_p = \Omega_p^{-1}$,  the wave excitation timescale, $\tau_w$, the gap opening timescale, $\tau_g$, and the viscous timescale, $\tau_\nu$. For realistic protoplanetary disks, $\tau_p < \tau_w < \tau_g < \tau_\nu$. In an isothermal, pressure supported disk without self-gravity the waves excited by the planet propagate as acoustic waves. The wave timescale can thus be approximated as $\tau_w \approx \mathcal{M} \tau_p$, where the Mach number $\mathcal{M} = (H/r)^{-1}$. The viscous timescale is $\tau_\nu = r^2 / \nu$. In this work, we use the $\alpha$-disk model for the kinematic viscosity, $\nu  = \alpha \left(H/r\right)^2 r^2 \Omega =  \mathcal{R}^{-1} r^2 \Omega$, so that $\tau_\nu = \mathcal{R} \Omega^{-1}$, where the Reynolds number is $\mathcal{R}= \mathcal{M}^2 / \alpha$ \citep{1973A&A....24..337S}. The gap opening timescale occurs somewhere between $\tau_w$ and $\tau_\nu$ and has been the subject of research for decades. We approximate this timescale as the viscous timescale at the location of the planet, $\tau_g = \mathcal{R} \tau_p$. 

For a given mean profile, the waves launched by the planet reach a steady state, $\partial_t \avg{ \Sigma' r^2 \Omega'} = 0$, on the wave timescale, $\tau_w$. On the gap opening timescale, $\tau_g$, the planet clears its coorbital region of gas. This produces piles of gas at the inner and outer edge of the gap. Finally, on the viscous timescale, $\tau_\nu$, these piles viscously spread out until the radial mass flux througout the  disk is spatially constant in radius. 




\subsection{Boundary Conditions} \label{sec:boundary_conditions}
We place our computational boundaries far from the planet so that $\Lambda_\text{dep} = 0$. This allows us to specify the boundary condition at both boundaries as,
\begin{equation} \label{eq:bc}
\left( \pderiv{r} \avg{r^2 \Omega} \right)^{-1} \pderiv{r} \left( -2 \pi r \nu \avg{\Sigma} r^2 \pderiv{r} \avg{\Omega} \right) = \dot{M} .
\end{equation}
At the outer boundary, we feed the disk at a given $\dot{M}$. While at the inner boundary, we extrapolate the surface density assuming a constant $\dot{M}$ across the boundary. These boundary conditions allow the disk to reach a steady state configuration where the inner radial mass flux approaches the forced mass flux at the outer boundary. Moreover, this allows the surface density to be enhanced at the outer boundary, rather than fixed to its initial value. 

%In order for our boundary conditions to be valid, we must ensure that there are no waves present at the boundaries. To this end, we utilize the wave killing boundaries of \cite{2006MNRAS.370..529D} in wave-killing regions near the inner and outer boundaries. In the inner wave-killing zone, we damp the radial velocity and surface density to their steady-state values that they would obtain if there was no planet in the disk. Note that, these values should also correspond to the steady-state values with the planet, see \S \ref{sec:steady_state_distribution}. In the outer wave-killing zone, we damp the radial velocity and surface density to their current azimuthal averages rather than their initial values. Figures \ref{fig:damping_late} and \ref{fig:damping_early} show the effects of different wave-killing prescriptions. 



\subsection{Steady State Distribution} \label{sec:steady_state_distribution}

In steady-state, the radial mass flux is spatially constant and the unsteady terms in \eqref{eq:mass_flux} are zero. In order to obtain a non-trivial steady-state solution, we fix the mass flux at the outer radius of the disk. We can rewrite \eqref{eq:mass_flux} in terms of the mass flux, $\dot{M} = -2 \pi r \avg{\Sigma v_r}$, the viscous angular momentum flux, $F_\nu = - 2 \pi r \langle \Sigma \rangle \nu r^3 \partial_r \langle \Omega \rangle$ and the specific angular momentum of the disk, $l = \langle r^2 \Omega \rangle$. For the steady-state solution we obtain,
\begin{equation}
\dot{M} = \frac{d F_\nu}{d l} - \left( \frac{d l}{d r} \right)^{-1} 2 \pi  r \Lambda_\text{dep} .
\end{equation}
We can integrate this equation using the inner boundary condition given by \eqref{eq:bc}, $F_\nu |_{l_i} = \dot{M} l_i$, to get the steady-state surface density distribution,
\begin{equation} \label{eq:ss_eqn}
\frac{\Sigma}{\Sigma_0} = \frac{F_\nu}{\dot{M} r^2 \Omega } = 1 + \frac{1}{\dot{M} r^2 \Omega } \int^r dr' \, 2 \pi r' \Lambda_\text{dep}(r'), 
\end{equation}
where the surface density in the absence of the planet satisfies, $\dot{M} = 3 \pi \nu \Sigma_0$. 

In the inner disk, where the deposited torque from the planet is zero, the disk is in its unperturbed state. Moving outwards, there is a drop in the surface density corresponding to the negative torque exerted on the disk. The depth and width of this gap is set by the balance of viscous and tidal torques and is controlled by the parameter $K = \left( M_p / M_\star \right)^2 \alpha^{-1} \left( H / r \right)^{-5}$ \citep{1986ApJ...309..846L, 2015ApJ...807L..11D,2015MNRAS.448..994K,2016PASJ...68...43K}. Exterior to the planet, the tidal torque changes sign and begins to remove the accumulated torque until the disk reaches its unperturbed surface density. This transition radius is given by, $ \int^{r_t} dr ' r' \Lambda_\text{dep}(r') = 0$. If there is an asymmetry between the inner and outer torques, the gas will begin to pile-up outside of the transition radius until the deposited torque finally falls to zero. This radius is determined by $\Lambda_\text{dep}(r \ge r_z) = 0$. At this radius, the surface density enhancement reaches a maximum given by,
\begin{equation}
\Sigma(r_z) = \Sigma_0(r_z) \left[ 1 + \frac{\Delta T}{\dot{M} r_z^2 \Omega(r_z)} \right],
\end{equation}
where the torque asymmetry is $ \Delta T = \int_{r_i}^{r_z} dr' \, 2 \pi r' \Lambda_\text{dep}(r')$.  Finally, exterior to $r_z$, the pile-up asymptotically approaches the unperturbed surface density as $(r^2 \Omega)^{-1}$. 
\begin{equation}
\Sigma(r > r_z) = \Sigma_0(r) \left[ 1 + \frac{\Delta T}{\dot{M} r^2 \Omega} \right] .
\end{equation}

We define the  magnitude of the pile-up, $\Pi$, as the ratio between the outer and inner surface densities where the deposited torque is non-zero, i.e between  $\Lambda_\text{dep}(r \le r_-) = 0$ and $\Lambda_\text{dep}(r \ge r_+) = 0$.
\begin{equation}
\Pi = \frac{\Sigma(r_+)}{\Sigma(r_-)} = \frac{\Sigma_0(r_+)}{\Sigma_0(r_-)} \left[ 1 + \frac{\Delta T}{\dot{M} r_+^2 \Omega_+} \right].
\end{equation}
We can simplify this expression by using the ratio of inner and outer viscosities, 
\begin{equation} \label{eq:pile_up}
\Pi = \frac{\nu_-}{\nu_+} \left[ 1 + \frac{\Delta T}{\dot{M} r_+^2 \Omega_+} \right].
\end{equation}
Assuming a constant aspect-ratio, Keplerian disk, the viscosity simplifies to, $\nu(r) = \alpha \left( H / r \right)^2 \sqrt{ G M_\star r}$. The pile-up factor now becomes,
\begin{equation}
\Pi = \sqrt{ \frac{ r_-}{r_+}} \left[ 1  + \frac{\Delta T}{\dot{M} \sqrt{ G M_\star r_+}} \right]
\end{equation}
We thus obtain a non-zero pile-up when, 
\begin{equation}
\frac{\Delta T}{\dot{M}} > \sqrt{G M_\star r_+} \left( \sqrt{ \frac{r_+}{r_-} } - 1 \right)  
\end{equation}
