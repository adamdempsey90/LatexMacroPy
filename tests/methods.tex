% !TEX root = ../main.tex



\subsection{Steady-State solution method} \label{sec:methods_ss}

Here we describe a solution method for quickly determining the steady-state profile of the disk. Our method makes use of the disparate evolution timescales for the axisymmetric flow (the viscous time) and for the waves launched by the planet (the orbital time). For a given axisymmetric state, $\avg{\Sigma}^n, \avg{v_r}^n, \avg{\Omega}^n$, at time $t_n$ the disk will quickly reach wave steady state (WSS), where the angular momentum density in the waves, $\avg{\Sigma' r^2 \Omega'}$ is constant over timescales much shorter than the viscous time. In WSS we can calculate the rate at which angular momentum is transferred from the waves into the axisymmetric disk as,
\begin{equation}
    \Lambda_\text{dep}^n = \Lambda_\text{ex}^n - \frac{1}{r} \pderiv{r} \left( r F_w^n \right), 
\end{equation}
Once we determine $\Lambda_\text{dep}$ from the current wave structure, we use \eqref{eq:ss_eqn} to determine the steady-state surface density, $\avg{\Sigma}^{ss}$, for the prescribed $\dot{M}$ at the outer boundary, and assume that $\avg{\Omega}^{ss} = \avg{\Omega}^n$. When evaluating the integral in \eqref{eq:ss_eqn}, we omit the contributions from the wave killing regions as the true deposited torque in these regions is uncertain. Once we have $\avg{\Sigma}^{ss}$, we set the radial velocity to, $\avg{v_r}^{ss} = -(\dot{M} - \avg{\Sigma'^n v_r{n,'}})/(2 \pi r \avg{\Sigma}^{ss})$. Finally, we repeat the above process for the next iteration, $\Sigma(r,\phi)^{n+1} = \avg{\Sigma}^{ss}$, etc. until convergence. We see from \eqref{eq:ss_eqn} that the surface density could fall below zero if $\int^r dr' \, 2\pi r' \Lambda_\text{dep}(r')  < - \dot{M} r^2 \Omega$. To prevent this, we enforce a floor on the steady-state surface density of $\Sigma(r)_\text{min} = \epsilon \Sigma_0(r)$, where $\epsilon$ is chosen small enough so that it does not prevent a deep gap from opening. The optimal value for $\epsilon$ should be slightly smaller than the true gap depth in steady state and so requires trial and error to determine.  A summary of this algorithm is,

\begin{enumerate}
\item Specify the outer boundary condition $\dot{M}_0$, the disk parameters, and the planet parameters.
\item Specify the initial distributions $\Sigma(r,\phi), v_r(r,\phi), v_\phi(r,\phi)$.
\item Evolve the 2D equations \eqref{eq:vel_eqn} \& \eqref{eq:dens_eqn} for $t_w$, where $t_w$ is the WSS time.
\item Calculate $\Lambda_\text{dep}$ by calculating the terms on the LHS of \eqref{eq:avg_ang}.
\item Calculate the steady-state surface density given $\Lambda_\text{dep}$ from \eqref{eq:ss_eqn}. If the surface density falls below some threshold, $\Sigma_\text{min} = \epsilon \Sigma_0$ then set $\avg{\Sigma}^{ss} = \Sigma_\text{min}$.
\item Set the 2D $\Sigma, v_r, v_\phi$ equal to
\begin{equation}
\Sigma(r,\phi) = \avg{\Sigma}^{ss} \qquad v_r(r,\phi) = - \frac{ \dot{M} - \avg{ \Sigma' v_r'}}{2 \pi r \avg{\Sigma}^{ss}} \qquad v_\phi(r,\phi) = \avg{v_\phi} .
\end{equation}
\item Evolve 2D equations for $t_w$ and repeat until the mass flux $\dot{M}$ is constant throughout the disk. 
\end{enumerate}

\subsection{\fargo}
In practice, our algorithm can be wrapped around any standard hydrodynamics code. For this work we use the community hydrodynamics code \fargo \citep{2016ApJS..223...11B}. \fargo solves \eqref{eq:vel_eqn} \& \eqref{eq:dens_eqn} on a staggered mesh where the density lies at the center of a cell and the velocities lie at the edges of the cell in their respective directions. \fargo solves the hydrodynamic equations via an operator splitting method, where the source and viscosity steps are implemented conservatively with finite differences. The transport step is implemented in a finite volume method where, instead of solving a Riemann problem, the wave speed is set to the fluid velocity in the appropriate direction. For simulating accretion disks, \fargo uses the fast advection algorithm of its predecessor to significantly increase the CFL limited timestep \citep{Masset:2000ku}. Because we are simulating planets with $M_p \ge M_J$ we implement the artificial viscosity module in \fargo to smooth the shocks that occur in the density wakes. To speed up our simulations, we run \fargo in its GPU mode on a cluster of NVIDIA K40 GPUs. Each simulation fits inside a single GPU. 

